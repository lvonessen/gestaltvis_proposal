\documentclass[12pt]{article}

\usepackage[tmargin=1in,bmargin=1in,lmargin=1in,rmargin=1in]{geometry}
\usepackage{setspace}
\usepackage{comment}
\usepackage{enumitem}
\usepackage{graphicx}
\usepackage{floatrow}

\setlist[enumerate]{itemsep=0mm}


\begin{document}
{\raggedleft 
	Laura Vonessen\\
	Honors Thesis Proposal\\ 
%	Due May 6 but got extension from Kate Schwartz of HC since studying abroad\\ 
%	SID: 23156536\\
}


\begin{comment}
Thesis proposal guidelines:
•	Clearly describe:
	o	Topic
			Why of interest to you and other scholars/researchers
			Relevance and importance
			Comparison to existing resources
	o	Form
			E.g. Research paper, empirical/observed research in lab, creative (poetry, etc.), other (community event, etc.), or combo
	o	How goals will be reached (presumably with goals themselves?)
	o	Depending on category, focus on:
			Science: questions/hypotheses, methods, theoretical/methodological framework
			•	Research methods/resources (e.g. library, archival, survey, interview…)
			Creative: Plan, importance to self, relevance to discipline (note: thesis itself will require written summary of product and process)
			•	Work involved
			•	Outcome/result
			•	Incorporation of analysis into final product
•	Include syllabus
	o	“aspirational learning outcomes” (now there’s a phrase that has never seen reality)
	o	(Expected) reading/lab/field work per semester
	o	(Expected) student/advisor meetings
	o	Expected work products (program plus paper, or something like?)
	o	Criteria for evaluation and grading
•	Prospectus bureaucracy
	o	1-2 pages plus possible bibliography
	o	Due May 6, 2015
	o	Form requires both student and advisor signatures (good luck)
	o	For humans, get research permissions
\end{comment}

\section{Proposal}

%OUTLINE:


Information visualization concerns itself with presenting data so as to facilitate interpretation of it. 
%When that data is in the form of objects and relationships, the traditional representation has been a graph, with nodes representing the objects and the edges between them representing their relationships.
%This still leaves much freedom in the placement of the nodes and arrangement of the edges, and 
This still leaves open how one should evaluate such a visualization. 
Should it be usability, but if so for what purpose?
Does attractiveness or memorability play a role?
%Thus the question of whether or not a visualization is ``pretty" at first seems only tangentially related.
% attractiveness, usability, what have you
Thus, the question of how aesthetics influences usability and user interaction with a visualization is a core question in this field. 
\\

A basic assumption of that question is that it is known what it means for something to be aesthetically pleasing. 
In fact, this is the topic of much research in the area of gestalt psychology. 
There, many principles have been identified and untersuched that help something to be perceived as a unified whole, or gestalt, more than the simple sum of its parts. For example, the principle of similarity states that people tend to perceive similar objects as parts of a whole (see Fig.~\ref{fig:sim}). It makes sense to build upon this research in asking how the individual gestalt principles contribute to a ``better" graph visualization.
\begin{comment}
\begin{figure}[h!]
	\floatbox[{\capbeside\thisfloatsetup{capbesideposition={left,center},capbesidewidth=5cm}}]{figure}[\FBwidth]
	{\caption{``The example [to the right] (containing 11 distinct objects) appears as a single unit because all of the shapes have similarity." (Spokane Falls GD)}\label{fig:sim}}
	{\includegraphics{similarity01.png}}
\end{figure}
\end{comment}
\\

Naturally, previous research in graph visualization will also play a role.



we have developed a series of hypotheses:
\begin{enumerate}[label=H\arabic*:]
	\item If the picture looks like a map is will be more appealing
	\item If the picture exhibits clear gestalt (closure, continuation, proximity) principles it will be more appealing
	\item If the picture has a high degree of familiarity (is more familiar) it enhances the pipeline
	\item Curved edges increase the attractiveness
	\item Data from studies of appeal can be analysed to determine underlying display properties (dimensions)
	\item Faces as circles (local measure of Aspect Ratio) means this will be more attractive
\end{enumerate}



\newpage
The goal:\\
To figure out links between how well a graph visualization corresponds to gestalt principles of psychology and its attractiveness/usability/what have you.

Important because:\\
Big Data! Woo. (Note, this is a name drop, not a point)\\
Well, information is useless unless you know what it is (can visualize it), and if it's a ``bad" visualization, that's bad, so you'd prefer a ``better" visualization. Only what's that?\\
What's out there?: well...

Information visualization concerns itself with presenting data so as to facilitate interpretation of it. 
Thus the question of whether or not a visualization is ``pretty" at first seems only tangentially related.
However, the question of how ``prettiness," or aesthetics, influences usability and user interaction with a visualization is a core question. 

Form:\\
Research paper(, framework for generating graphs as for survey?)

The plan:\\
Survey of lit, survey of people, analysis of results\\


In final form, include:\\
questions/hypotheses, methods, theoretical/methodological framework, resources (e.g. library, archival, survey, interview...)


\begin{comment}
The Honors College recommends a ``syllabus" with the proposal

\section{Syllabus}
\subsection{Aspirational learning outcomes}
Gain skills in:
\begin{enumerate}
	\item Research methods/processes
	\item Reading papers
	\item Designing/carrying out a survey
	\item Compiling results into a paper
	\item Steps involved in publishing a paper
\end{enumerate}
\subsection{Expected work}
Phase 1
\begin{enumerate}
	\item Survey of gestalt/graph vis papers (Scan papers, decide most relevant, read those more in depth, discuss with Stephen and compare)
	\item Compile what's there/what's relevant
\end{enumerate}
Phase 2
\begin{enumerate}
	\item Use that to update Stephen's hypotheses/survey goals
	\item Program visualizations, etc, to spread over factors want to test
	\item Design survey to gather that info
\end{enumerate}
Phase 3
\begin{enumerate}
	\item Conduct pre-survey as test run and make any necessary modifications (make sure likely to collect the data we're looking for)
	\item Run full survey
\end{enumerate}
Phase 4
\begin{enumerate}
	\item Analyze data (correlations, confoundings, etc)
	\item Come to conclusions (yes/no to Hs?)
	\item Write up paper/make pretty figures
\end{enumerate}
\subsection{Expected meetings}
\begin{enumerate}
	\item At least once a week on skype?
	\item Plus email
	\item Plus anything else that comes up?
\end{enumerate}
\subsection{Expected product(s)}
\begin{enumerate}
	\item Survey of existing lit(?)
	\item Framework to generate survey graphs(?)
	\item Data from survey
	\item Research paper
\end{enumerate}
\subsection{Criteria for evaluation}
Um...I don't know?
\end{comment}

\end{document}