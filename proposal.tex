\documentclass[12pt, twocolumn]{article}

\usepackage[tmargin=.8in,bmargin=.8in,lmargin=1in,rmargin=1in]{geometry}
\usepackage{setspace}
\usepackage{comment}
\usepackage{enumitem}
\usepackage{graphicx}
\usepackage{floatrow}

\setlist[enumerate]{itemsep=0mm}



\title{\vspace{-.5in}Gestalt visualization proposal}
%\title{\vspace{-1cm}Honors Thesis Proposal: Gestalt Visualization}
\author{} %Laura Vonessen, \today}
%\date{Due May 6, but received extension since studying abroad}
\date{\vspace{-.7in}}


\begin{document}
%{\raggedleft 
%	Laura Vonessen\\
%	Honors Thesis Proposal\\ 
%	Due May 6 but got extension from Kate Schwartz of HC since studying abroad\\ 
%	SID: 23156536\\
%}
\maketitle


\thispagestyle{empty}

\begin{comment}
Thesis proposal guidelines:
•	Clearly describe:
	o	Topic
			Why of interest to you and other scholars/researchers
			Relevance and importance
			Comparison to existing resources
	o	Form
			E.g. Research paper, empirical/observed research in lab, creative (poetry, etc.), other (community event, etc.), or combo
	o	How goals will be reached (presumably with goals themselves?)
	o	Depending on category, focus on:
			Science: questions/hypotheses, methods, theoretical/methodological framework
			•	Research methods/resources (e.g. library, archival, survey, interview…)
			Creative: Plan, importance to self, relevance to discipline (note: thesis itself will require written summary of product and process)
			•	Work involved
			•	Outcome/result
			•	Incorporation of analysis into final product
•	Include syllabus
	o	“aspirational learning outcomes” (now there’s a phrase that has never seen reality)
	o	(Expected) reading/lab/field work per semester
	o	(Expected) student/advisor meetings
	o	Expected work products (program plus paper, or something like?)
	o	Criteria for evaluation and grading
•	Prospectus bureaucracy
	o	1-2 pages plus possible bibliography
	o	Due May 6, 2015
	o	Form requires both student and advisor signatures (good luck)
	o	For humans, get research permissions
\end{comment}

%\section{Proposal}

%OUTLINE:


Information visualization concerns itself with presenting data so as to facilitate interpretation of it. 
%When that data is in the form of objects and relationships, the traditional representation has been a graph, with nodes representing the objects and the edges between them representing their relationships.
%This still leaves much freedom in the placement of the nodes and arrangement of the edges, and 
In graph visualization, this information comes in the form of a graph, with the information typically organized as objects (nodes) and relationships between them (edges).
This still leaves open the evaluation of such a visualization. 
Should the main factor be usability, but if so then usability for what purpose?
Do attractiveness and memorability play a role?
%Thus the question of whether or not a visualization is ``pretty" at first seems only tangentially related.
% attractiveness, usability, what have you
Thus, the question of how aesthetics influences user interactions with and reactions to a given visualization is a core question in this field. 
\\

A basic assumption of that question is that it is known what it means for something to be aesthetically pleasing. 
In fact, this is the topic of much research in the area of gestalt psychology. 
There, many principles have been identified and investigated that help something to be perceived as a unified whole, or gestalt, more than the simple sum of its parts. 
For example, the principle of similarity states that people tend to perceive similar objects as parts of a whole, or the dissimilar object among a set of similar ones as the focus (see Fig.~\ref{fig:sim}). 
It makes sense to build upon this research in asking how the individual gestalt principles contribute to a ``better" graph visualization.
Our hypothesis is that gestalt principles contribute to both aesthetic appeal and analytic performance in force-directed layout networks.
\begin{figure}[b]
	%	\floatbox[{\capbeside\thisfloatsetup{capbesideposition={left,center},capbesidewidth=5cm}}]{figure}[\FBwidth]{
	\caption{\small ``The figure on the far right becomes a focal point because it is dissimilar to the other shapes." (reproduced from Spokane Falls GD, cite!)}
	\label{fig:sim}
	%	}{
	\includegraphics[width=.9\linewidth]{anomaly01.png}
	%	}
\end{figure}
%\begin{figure}[h!]
%%	\floatbox[{\capbeside\thisfloatsetup{capbesideposition={left,center},capbesidewidth=5cm}}]{figure}[\FBwidth]{
%	\caption{\small``The example above (containing 11 distinct objects) appears as a \emph{single unit} because all of the shapes have \emph{similarity}." (reproduced from Spokane Falls GD, cite!)}
%	\label{fig:sim}
%%	}{
%	\includegraphics[width=5cm]{similarity01.png}
%%	}
%\end{figure}
\\

To evaluate this hypothesis, we must first conduct a survey of the literature in order to match gestalt principles to the graph layout heuristics, such as minimizing edge crossings, which have tended to be the focus of previous graph visualization research. 
%Thus, an extensive survey of the literature must be done first to match heuristics to the gestalt principles they most closely align with in order to generate the graphs we will show to subjects.
We will perform this survey of the literature with the aid of the SurVis system, which allows sources to be tagged and searched in an intuitive manner (cite).
Once the matchings between heuristics and gestalt principles have been identified, we plan to make a poster showing these and submit it to GD. 
At that point, we will have the information we need to choose the gestalt principles to test in our user study, and we will know which graph heuristics to use to generate graphs which do or do not follow these principles.
\\
(Add something about making the literature survey available, or is it just for us?)
\\


% The user study
The user study will present subjects with graph embeddings following these  gestalt principles to varying degrees.
%We have yet to decide whether the user study will be conducted in person or, for example, via Amazon's Mechanical Turk. 
To ensure our answers are minimally biased, users will be asked to perform tasks and provide preferences instead of merely telling us which graphs are most aesthetically pleasing and ``useful."
Thus, each graph will have several questions associated with it, and as we would prefer a within user 
% In between, participants see only one set of conditions---thus, you need more participants.
% In within, participants see all sets of conditions---thus, can make comparisons for how this person reacted to the different conditions. Requires shorter question(s/-sets) because can’t tire them out.
study, the length and thus the number of test factors will need to be limited in order to retain the interest of our subjects. 
Fortunately, the principle/heuristic table will help us to do so.
Finally, we will need to translate the responses to our questions into answers as straightforward as possible on which graphs are more aesthetically appealing and/or usable, and thus whether and which gestalt principles have the most impact. 
%The generation of these graphs will require an extensive literature survey to 











% Actually I think the following was in relation to something else:
%we have developed a series of hypotheses:
%\begin{enumerate}[label=H\arabic*:]
%	\item If the picture looks like a map is will be more appealing
%	\item If the picture exhibits clear gestalt (closure, continuation, proximity) principles it will be more appealing
%	\item If the picture has a high degree of familiarity (is more familiar) it enhances the pipeline
%	\item Curved edges increase the attractiveness
%	\item Data from studies of appeal can be analysed to determine underlying display properties (dimensions)
%	\item Faces as circles (local measure of Aspect Ratio) means this will be more attractive
%\end{enumerate}



%\newpage
%The goal:\\
%To figure out links between how well a graph visualization corresponds to gestalt principles of psychology and its attractiveness/usability/what have you.
%
%Important because:\\
%Big Data! Woo. (Note, this is a name drop, not a point)\\
%Well, information is useless unless you know what it is (can visualize it), and if it's a ``bad" visualization, that's bad, so you'd prefer a ``better" visualization. Only what's that?\\
%What's out there?: well...
%
%Information visualization concerns itself with presenting data so as to facilitate interpretation of it. 
%Thus the question of whether or not a visualization is ``pretty" at first seems only tangentially related.
%However, the question of how ``prettiness," or aesthetics, influences usability and user interaction with a visualization is a core question. 
%
%Form:\\
%Research paper(, framework for generating graphs as for survey?)
%
%The plan:\\
%Survey of lit, survey of people, analysis of results\\
%
%
%In final form, include:\\
%questions/hypotheses, methods, theoretical/methodological framework, resources (e.g. library, archival, survey, interview...)


\newpage
%\begin{comment}
% % % % % % % % % % % % % % % % % % % %
% % The Honors College recommends a ``syllabus" with the proposal
% % % % % % % % % % % % % % % % % % % %
\section{Syllabus}
\subsection{Aspirational learning outcomes}
% I am not at fault for the ridiculous title of this subsection
% Blame the honors college
Gain skills in:
\begin{enumerate}
	\item Research methods/processes
	\item Reading papers
	\item Designing/carrying out a survey
	\item Compiling results into a paper
	\item Steps involved in publishing a paper
\end{enumerate}
\subsection{Expected work}
Phase 1
\begin{enumerate}
	\item Survey of gestalt/graph vis papers (Scan papers, decide most relevant, read those more in depth, discuss with Stephen and compare)
	\item Use StarVis to compile/organize what's there/what's relevant
	\item Update gestalt-heuristics matchings
\end{enumerate}
Phase 2
\begin{enumerate}
	\item Make a poster
	\item Use phase 1 work to update hypotheses/survey goals
	\item Program visualizations, etc, to spread over factors want to test
	\item Design survey to gather that info
\end{enumerate}
Phase 3
\begin{enumerate}
	\item Conduct prototype study as test run and make any necessary modifications (make sure likely to collect the data we're looking for)
	\item Run full user study
\end{enumerate}
Phase 4
\begin{enumerate}
	\item Analyze data (correlations, confoundings, etc)
	\item Come to conclusions (yes/no to Hs?)
	\item Write up paper/make pretty figures
\end{enumerate}
\subsection{Expected meetings}
\begin{enumerate}
	\item At least once a week (on skype, after no longer in same country)
	\item Plus email
	\item Plus anything else that comes up?
\end{enumerate}
\subsection{Expected product(s)}
\begin{enumerate}
	\item Survey of existing lit(?)
	\item Framework to generate survey graphs(?)
	\item Data from survey
	\item Research paper
\end{enumerate}
\subsection{Criteria for evaluation}
\begin{enumerate}
	\item Work completed with reasonable thoroughness, best practices
	\item Completed on reasonable timeline (Note: phases not entirely sequential), for example:
	\begin{enumerate}
		\item Phase 1 by mid July (3 weeks)
		\item Phase 2 by first week of August (3 more)
		\item Phase 3 by early Sept (4 weeks)
		\item Phase 4 by end Sept (2.5?) (two possible conference deadlines)
	\end{enumerate}
	\item ???
\end{enumerate}
%\end{comment}

\end{document}