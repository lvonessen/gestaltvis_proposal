\documentclass[12pt]{article}

\usepackage[tmargin=1in,bmargin=1in,lmargin=1in,rmargin=1in]{geometry}
\usepackage{setspace}
\usepackage{comment}
\usepackage{enumitem}

\setlist[enumerate]{itemsep=0mm}


\begin{document}
{\raggedleft 
	Laura Vonessen\\
	Honors Thesis Proposal\\ 
%	Due May 6 but got extension from Kate Schwartz of HC since studying abroad\\ 
%	SID: 23156536\\
}


\begin{comment}
Thesis proposal guidelines:
•	Clearly describe:
	o	Topic
			Why of interest to you and other scholars/researchers
			Relevance and importance
			Comparison to existing resources
	o	Form
			E.g. Research paper, empirical/observed research in lab, creative (poetry, etc.), other (community event, etc.), or combo
	o	How goals will be reached (presumably with goals themselves?)
	o	Depending on category, focus on:
			Science: questions/hypotheses, methods, theoretical/methodological framework
			•	Research methods/resources (e.g. library, archival, survey, interview…)
			Creative: Plan, importance to self, relevance to discipline (note: thesis itself will require written summary of product and process)
			•	Work involved
			•	Outcome/result
			•	Incorporation of analysis into final product
•	Include syllabus
	o	“aspirational learning outcomes” (now there’s a phrase that has never seen reality)
	o	(Expected) reading/lab/field work per semester
	o	(Expected) student/advisor meetings
	o	Expected work products (program plus paper, or something like?)
	o	Criteria for evaluation and grading
•	Prospectus bureaucracy
	o	1-2 pages plus possible bibliography
	o	Due May 6, 2015
	o	Form requires both student and advisor signatures (good luck)
	o	For humans, get research permissions
\end{comment}

\section{Proposal}

OUTLINE:

The goal:\\
To figure out links between how well a graph visualization corresponds to gestalt principles of psychology and its attractiveness/usability/what have you.

Important because:\\
Big Data! Woo. (Note, this is a name drop, not a point)\\
Well, information is useless unless you know what it is (can visualize it), and if it's a ``bad" visualization, that's bad, so you'd prefer a ``better" visualization. Only what's that?\\
What's out there?: well...

Form:\\
Research paper(, framework for generating graphs as for survey?)

The plan:\\
Survey of lit, survey of people, analysis of results\\


In final form, include:\\
questions/hypotheses, methods, theoretical/methodological framework, resources (e.g. library, archival, survey, interview...)


\begin{comment}
The Honors College recommends a ``syllabus" with the proposal

\section{Syllabus}
\subsection{Aspirational learning outcomes}
Gain skills in:
\begin{enumerate}
	\item Research methods/processes
	\item Reading papers
	\item Designing/carrying out a survey
	\item Compiling results into a paper
	\item Steps involved in publishing a paper
\end{enumerate}
\subsection{Expected work}
Phase 1
\begin{enumerate}
	\item Survey of gestalt/graph vis papers (Scan papers, decide most relevant, read those more in depth, discuss with Stephen and compare)
	\item Compile what's there/what's relevant
\end{enumerate}
Phase 2
\begin{enumerate}
	\item Use that to update Stephen's hypotheses/survey goals
	\item Program visualizations, etc, to spread over factors want to test
	\item Design survey to gather that info
\end{enumerate}
Phase 3
\begin{enumerate}
	\item Conduct pre-survey as test run and make any necessary modifications (make sure likely to collect the data we're looking for)
	\item Run full survey
\end{enumerate}
Phase 4
\begin{enumerate}
	\item Analyze data (correlations, confoundings, etc)
	\item Come to conclusions (yes/no to Hs?)
	\item Write up paper/make pretty figures
\end{enumerate}
\subsection{Expected meetings}
\begin{enumerate}
	\item At least once a week on skype?
	\item Plus email
	\item Plus anything else that comes up?
\end{enumerate}
\subsection{Expected product(s)}
\begin{enumerate}
	\item Survey of existing lit(?)
	\item Framework to generate survey graphs(?)
	\item Data from survey
	\item Research paper
\end{enumerate}
\subsection{Criteria for evaluation}
Um...I don't know?
\end{comment}

\end{document}